\documentclass[a4paper,10pt]{article} % Default is not A4 but US letter size, 10pt default
%\usepackage{extsizes} % Improves range from (pt) 10,11,12 to 8,9,10,11,12,14,17,20
%\usepackage{graphicx} % Used to import external graphics e.g. images
\usepackage{amsmath, amssymb, amsfonts} % Extend alphabet of mathematical symbols and provide improved equation environment
\usepackage[top=3cm, bottom=4cm, left=3cm, right=3cm]{geometry} % Used to define page dimensions
\usepackage{hyperref} % Enables hyperlinks (online links + referencing), use [hidelinks] optionallt
\usepackage{xcolor} % Sets colour for hyperlinks instead of ugly boxes
\hypersetup{
    colorlinks,
    linkcolor={red!50!black},
    citecolor={blue!50!black},
    urlcolor={blue!80!black}
}

\begin{document}
\thispagestyle{empty}
\newcommand{\Rule}{\rule{\textwidth}{1mm}}
\begin{center}
\ \\
\vspace{10mm}

\huge{\textsc{University of Birmingham}} \par
\vspace{7mm}
\Large{\textsc{College of Engineering and Physical Sciences}}\par
\vspace{2mm}
\Large{\textsc{School of Computer Science}}\par
\vspace{15mm}

\Rule
\vspace{11mm}
\huge\textbf{Building a Falcon 9 Rocket Model}\par
\vspace{9mm}
\Rule
\par
\ \\
\vspace{2mm}
\LARGE{\textbf{Undergraduate project}}\par
\vspace{4mm}
\large{in various fields including Computer Science, Electronics,\\ Robotics, Mechanical and Aerospace Engineering}


\vspace{12mm}

\textsc{Performed by Computer Science BSc 1st Years} \ \\
\vspace{3mm}
Jonas \textsc{Schäfer} - 1944365 \par
Phan \textsc{Minh Cuong} - 0000000
\vspace{5mm}
\\
Project begin: 22.02.2019 \\
Project completion: 00.00.2019
\vspace{15mm}

Birmingham, \today\par

\vspace{30mm}

\end{center}
\newpage
\tableofcontents

\newpage

\section{Basic information}

\subsection{The goal of the project}

The final goal is to build a functional SpaceX Falcon 9 rocket model.\\
The model is planned to use Thrust Vectoring Control (TVC), the current
principle of choice is a motorised gimbal that stabilises the
rocket by moving the rocket engine/nozzle and therefore adjust the angle
of the engine thrust. The TVC will be controlled over a programmed
Arduino micro-controller that reads sensor data from multiple sensors
and returns appropriate behaviour instructions to the gimbal system.
While current plans don't go as far as landing the rocket, we consider
different approaches how we could achieve this for an extended future project (e.g. by being able to
throttle the engine or reigniting the propulsion system used).

\subsection{Development Plan}

Obviously this is a very ambitious project and many small steps must be
taken before we can take on the actual challenge of building the Falcon 9 model rocket. Below is the concept by
which we plan to progress. The sub-projects will increase in complexity and difficulty and are supposed to slowly
teach us all necessary aspects to take on the rocket design for the Falcon 9 model rocket.

\subsubsection*{Project 1: Getting to know the 3-Axis Accelerometer and Gyroscope}
To gain a better understanding of the usage of the accelerometer and
gyroscope (precisely the \href{https://playground.arduino.cc/Main/MPU-6050}{\texttt{GY-521 3 Axis
Gyroscope + Accelerometer Module MPU-6050}}) we plan to do some basic
testing of its functionalities using the Arduino Uno as the receiving
and processing board.

\subsubsection*{Project 2: Getting to know basic Rocketry by building a model rocket}
To get a better view on the practical side of things, we then plan to build a
basic rocket and use an \href{https://www.estesrockets.com/rockets/engines}{industry-standard
certified \texttt{Estes rocket motor}} to launch it. We hope to understand
practical issues we could encounter and important concepts we need to
research further to enable our rocket to fly as stable as possible while
remaining as close to an actual practical rocket design as possible (reduced use of of fins
inducing atmospheric drag).

\subsubsection*{Project 3: Building a 2D stabilising device}
Using the knowledge from Project 1 we plan to build a device that is able
to stabilise itself and compensate for exterior forces acting on it using the
accelerometer and gyroscope data. After this project we are as prepared as we can be
to develop a stabilising system in a 3D operating vehicle.

\subsubsection*{Project 4: Radio Control}
Before moving on to Project 5 we could introduce the concept of radio
control which would enable us to control the rocket remotely. Prior to launching a self-stabilising
rocket, a connection system between the rocket and Ground Control should
be established in order to guarantee safety and enable controlled
remote takeoff and landing.

\newpage

\subsubsection*{Project 5: Introducing a gimbal system and using a propeller for propulsion}
In a way, we take a step back and turn our rocket into something close to a monocopter. We exchange the hard-to-control rocket motor with an electric motor with propellers. 
This way we will be able to introduce the gimbaling system without having to worry about the complex rocket engine
thrust and we will be able to throttle the rotor to easier test the
gimbaling system with less damages and potential safety issues. The
gimbaling system will bring all the challenges the final project version
will encounter using rocket engines but is safer and easier to handle.

\subsubsection*{Final Project: Realizing the Falcon 9 Rocket Model}
The final step will be to re-introduce the rocket engine and optimise the
gimbal system to successfully stabilise the rocket during flight. If we
reach this point, further project ideas on re-ignitable rocket engines
or thrust-control to enable a takeoff and landing procedure could
follow.

\newpage
\section{The Learning Process}

\subsection{Getting to know the 3-Axis Accelerometer and Gyroscope}

\newpage
\subsection{Getting to know basic Rocketry by building a model rocket}

\newpage
\subsection{Building a 2D stabilising device}

\newpage
\subsection{Radio Control}

\newpage
\subsection{Introducing a gimbal system and using a propeller
for propulsion}

\newpage
\section{Realizing the Falcon 9 Rocket Model}


\end{document}
